\documentclass[a4paper]{ltjsarticle}
\usepackage{luatexja}
\usepackage[T1]{fontenc}
\usepackage{lmodern}
\usepackage[no-math]{fontspec}
\usepackage[unicode,hidelinks,pdfusetitle]{hyperref}
\usepackage{tcolorbox}
\usepackage{amsmath,amssymb}
\usepackage{bm}
\usepackage{latexsym}
\usepackage{mathrsfs}
\usepackage{braket}
\usepackage{graphicx}
\usepackage{tikz,scsnowman}
\usepackage{amsthm}
\def\qed{\hfill $\Box$}
\newtheorem{dfn}{定義}
\newtheorem{thm}{定理}
\newcommand{\triset}[3]{\overset{#3}{\underset{#1}{#2}}}
\begin{document}
\title{unionのDFAにおける証明の補足}
\date{}
\author{}
\maketitle
\begin{abstract}
p.45-p.47における証明において、alphabetが同じという仮定があったので、その仮定なしの一般化した証明を載せておきます.
\end{abstract}

\begin{tcolorbox}[title = Theorem 1.25]
  The class of regular languages is closed under the union operation
\end{tcolorbox}

\section*{証明}
$A_1,A_2$を任意にとってきた正規言語とします.

$A_1$を認識するDFAを$M_1$,$A_2$を認識するDFAを$M_2$とします.

$M_1=(Q_1,\Sigma_1,\delta_1,q_1,F_1), M_2=(Q_2,\Sigma_2,\delta_2,q_2,F_2)$であるとします.

このとき, $A_1 \cup A_2$を認識するDFA$M$を次のように構成します.

$M=((Q_1\cup\Set{s})\times (Q_2\cup\Set{s}),\Sigma_1\cup\Sigma_2,\delta,(q_1,q_2),F)$とします. ただし、次に述べる条件を$\delta,F,s$は満たします.

1.$s$は$Q_1$にも$Q_2$にも含まれない要素です.

2.$\delta\colon ((Q_1\cup\Set{s})\times (Q_2\cup\Set{s}))\times (\Sigma_1\cup\Sigma_2) \rightarrow ((Q_1\cup\Set{s})\times (Q_2\cup\Set{s}))$は次のように定義します.

そのために$\delta_1,\delta_2$を次のように拡張します.

$\delta_1'\colon (Q_1\cup\Set{s})\times(\Sigma_1\cup\Sigma_2) \rightarrow (Q_1\cup\Set{s})を
\delta_1'(r_1,a)=
\begin{cases}
  \delta_1(r_1,a) & r_1\in Q_1,a\in\Sigma_1\\
  s & otherwise\\
\end{cases}$
と定めます.

同様に
$\delta_2'\colon (Q_2\cup\Set{s})\times(\Sigma_1\cup\Sigma_2) \rightarrow (Q_2\cup\Set{s})を
\delta_2'(r_2,a)=
\begin{cases}
  \delta_2(r_2,a) & r_2\in Q_2,a\in\Sigma_2\\
  s & otherwise\\
\end{cases}$
と定めます.

この$\delta_1',\delta_2'$を利用して, $\delta((r_1,r_2),a) = (\delta_1'(r_1,a),\delta_2'(r_2,a))$として$\delta$を定めます.

3.$F=\Set{(r_1,r_2)|(r_1\in F_1 \,or\, r_2\in F_2)}$.

以上の3条件を満たす$M$は$A_1\cup A_2$を認識する言語になっています.

\end{document}